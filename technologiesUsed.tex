\chapter{Technologies Used}



\section{Tesseract Optical Character Recognition Engine:}
An open source OCR engine licensed under Apache License, Version 2.0.\footnotemark~It is one of the most accurate open source OCR engines currently available.


%\section{TensorFlow:}
%	An open source library of programing functions for machine learning. Licensed under Apache License, Version 2.0\footnotemark[\value{footnote}].

\footnotetext{https://www.apache.org/licenses/LICENSE-2.0.txt}

\section{OpenCV:}
An open source library of programing functions mainly aimed at real-time computer vision. Licensed under BSD license.\footnotemark

\section{Flask:}
Flask is a lightweight Python web framework upon which PRAHVI's backend is built.

\section{Nginx:}
Nginx is a proxy server technology used to deploy our Flask web server and expose it via the internet.

%\section{NumPy:}
%	An extension (numerical and scientific library) to the Python programming language, it adds support for large, multidimensional arrays and matrices. Licensed under BSD-new license\footnotemark[\value{footnote}].


%\section{scikit-learn:}
%	A free machine learning library for Python programming language, designed to interoperate with NumPy and SciPy. Licensed under BSD license\footnotemark[\value{footnote}]. 


%\section{scikit-image:}
%	An image processing library for Python programming language, designed to interoperate with NumPy and SciPy. Licensed under BSD license\footnotemark[\value{footnote}].

\footnotetext{https://en.wikipedia.org/wiki/BSD\_licenses}