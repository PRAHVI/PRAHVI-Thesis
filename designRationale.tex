\chapter{Design Rationale}

	\section{Architecture}
		\begin{itemize}
			\item Hardware
			\begin{itemize}
				\item Wired Headset Device Form Factor
				\begin{itemize}
					\item Although many consumer electronics are moving to wireless form factors, the wired headset form factor was chosen to deliver a better experience that fits this unique set of users' needs. In particular, we focused on the areas of latency and usability. The wearable form factor was chosen because it best fit the use cases that users are presented with when translating text. This means that the headset is readily available and in-position for the user to translate text upon request. Utilizing a wired form factor ensures that there is no latency or ambiguity in setting up the connection and communicating from the device to the smartphone. A wired form-factor also potentially eliminates the need for the user to maintain knowledge of a secondary battery or configure their device to pair with the smartphone.
				\end{itemize}
				\item Haptic and Audible User interface
				\begin{itemize}
					\item The smartphone application that accompanies the device communicates with the user entirely through haptic and audible feedback. The user interface is deliberately "stripped-down" to its singular component, the gesture area. The user double-taps to begin the translation process, single-taps to pause or start a dictation, and swipe to navigate. This product must appeal to people with a sliding scale of blindness, some with varying conditions that warrant a user interface that can appeal to the lowest common denominator.
				\end{itemize}
			\end{itemize}


			\item Software
			\begin{itemize}
				\item Image preprocessing before Optical Character Recognition
				\begin{itemize}
					\item The current Tesseract OCR Engine operates poorly with images that contain skewed text or visual artifacts. To improve on the overall accuracy of the system, preprocessing is employed to detect blur and visual artifacts while deskewing the image as best as possible to improve performance of the OCR engine.
				\end{itemize}
			\end{itemize}
		\end{itemize}
\pagebreak


	\section{Technologies Used}
		\begin{itemize}
			\item Tesseract Optical Character Recognition Engine version 4
			An open source OCR engine licensed under Apache License, Version 2.0.\footnotemark~It is one of the most accurate open source OCR engines currently available.
			
			\footnotetext{https://www.apache.org/licenses/LICENSE-2.0.txt}
			\begin{itemize}
				\item Currently, the most accurate open source solution to optical character recognition
				\item Using state of the art machine learning models for word recognition
				\item Highly documented with academic papers written about it’s architecture
				\item Constantly being developed by a community of developers, so if we run into problems we have a community to ask questions to
				\item Has both a C (python wrapper) and a C++ API
				\item Supported by Google
			\end{itemize}

			\item OpenCV
			An open source library of programming functions mainly aimed at real-time computer vision. Licensed under BSD license.\footnotemark
			\footnotetext{https://en.wikipedia.org/wiki/BSD\_licenses}
			\begin{itemize}
				\item De Facto standard software libraries for computer vision
				\item Lots of developer support
				\item Open source
				\item All of its data structures are compatible with Tesseract's API
			\end{itemize}
			
			\item Flask
			Flask is a lightweight Python web framework upon which PRAHVI's backend is built.
			\begin{itemize}
				\item Web client framework with the smallest learning curve
				\item Fast and intuitive web API development
				\item Open Source
			\end{itemize}

			\item Nginx
			Nginx is a proxy server technology used to deploy our Flask web server and expose it via the internet.
			\begin{itemize}
				\item Allows us to deploy our Flask application onto the web
				\item Seamless integration with Flask
				\item Open Source
			\end{itemize}

		\end{itemize}