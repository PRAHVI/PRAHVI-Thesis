\chapter{Requirements}
The Requirements section presents a categorized and itemized list of project requirements. Categories include Functional and Non-functional Requirements and Design Constraints. Functional requirements define what must be done, while non-functional requirements define the manner in which the functional requirements need to be achieved. Both categories have sub-categories, determined by the importance of a given requirement. Design Constraints are similar to non-functional requirements but constrain the solution and not the problem.
\section{Functional Requirements:}
\begin{itemize}
\item Critical
	\begin{itemize}
	\item have sensors to communicate with the user and detect the user’s surroundings
	\item communicate with the user through haptic feedback and voice dictation
    \end{itemize}
\item Recommended
	\begin{itemize}
	\item have a learning model to tag specific instances of text and symbols and improve overall recognition
	\end{itemize}
\end{itemize}

\pagebreak

\section{Non-functional Requirements:}
\begin{itemize}
\item Critical
	\begin{itemize}
	\item easy and intuitive to recognize text in the user’s environment and dictate it to the user
	\item light and untethered from a large computing system
	\item conform to the Federal Communications Commission guidelines on wearable devices\footnote{https://www.fcc.gov/general/ingestibles-wearables-and-embeddables}
    \end{itemize}
\item Recommended
	\begin{itemize}
	\item maintainable for future usage and/or upgrades
	\end{itemize}
\item Suggested
    \begin{itemize}
	\item generally aesthetically pleasing so that it does not draw too much attention from the user’s surroundings
	\end{itemize}
\end{itemize}

\section{Design Constraints}
\begin{itemize}
\item The main device is a wearable headset whose hardware is self-contained
\item The main device’s main communication with the outside world is through a smartphone
\item The main device communicates primarily through sound and touch, and not through vision
\end{itemize}
