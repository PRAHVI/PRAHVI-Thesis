\chapter{Requirements}
The Requirements section presents a categorized and itemized list of project requirements. Categories include Functional and Non-functional Requirements and Design Constraints. Functional requirements define what must be done, while non-functional requirements define the manner in which the functional requirements need to be achieved. Both categories have sub-categories, determined by the importance of a given requirement. Design Constraints are similar to non-functional requirements but constrain the solution and not the problem.
\section{Functional Requirements:}
\begin{itemize}
\item Critical
	\begin{itemize}
	\item This system must have a visual sensor to detect text in users field of view. This ensures the device is able to recognize the text the user needs translated.
	\item This system must communicate with the user through haptic feedback and voice dictation. To be usable for individuals with visual disabilities, this device must use forms of feedback and interaction that do not involve sight.
    \end{itemize}
\item Recommended
	\begin{itemize}
	\item This system will have a learning model to tag specific instances of text and symbols. This allows the system to sustainably and effectively improve its overall text and symbol recognition over time.
	\end{itemize}
\item Specifications
	\begin{itemize}
	\item This system must process its image to text translation in a duration of 10 seconds or less.
	\item This system be able to recognize text on a 8.5"x11" page within a field of view of 90 degrees from 2 feet away.
	\item This system must be able to recognize font sizes of 10pt to 50pt within the parameters specified above.
	\item This system must be able to translate images to text with a word-accuracy of 80%.
	\item This system must be able to translate structured paragraph style text articles without tabular text or images. 
	\end{itemize}
\end{itemize}

\pagebreak

\section{Non-functional Requirements:}
\begin{itemize}
\item Critical
	\begin{itemize}
	\item This system must be easy and intuitive to recognize text in the user's environment and dictate it to the user. The target user should be able to use the system with minimal technical knowledge and/or training to ensure its effectiveness.
	\item This system must be compliant with the Federal Communications Commission guidelines on wearable devices\footnote{https://www.fcc.gov/general/ingestibles-wearables-and-embeddables}
	\item This system must be affordable for users, around or less than \$100.
    \end{itemize}
\item Recommended
	\begin{itemize}
	\item This system must be maintainable for future usage and/or upgrades. This means that the system's range of capabilities can be expanded upon in the future.
	\item This system must be aesthetically unobtrusive for public consumption. As a device meant to help users more easily integrate into their social contexts, this device should provide the aforementioned capabilities without drawing unwarranted public attention to the user.
	\item This system must be lightweight and minimal, less than 36 grams and no larger than the footprint of typical large sunglasses. In order to be used daily, the hardware of the device must only add minimal friction to the user's lifestyle, meaning that the device cannot have extraneous parts or weight that would hinder usage.
	\end{itemize}
\end{itemize}

\section{Design Constraints}
\begin{itemize}
\item The main device is a wearable headset whose hardware is self-contained
\item The main device's main communication with the outside world is through a smartphone
\item The main device communicates primarily through sound and touch, and not through vision
\item This system must be untethered from a large computing system.
\end{itemize}
